\mychapter{Column Generation and Branch-and-Price \wip}

Column generation and branch-and-price are advanced techniques for solving
large-scale integer programming problems. This chapter introduces the
fundamental concepts and demonstrates how to implement these techniques
using~\textsf{idol}. For a detailed treatment of the theoretical foundations,
we refer the reader to the books by~\textcite{desrosiers2024branch}
and~\textcite{uchoa2024optimizing}.

The chapter is organized as follows. In Section~\ref{sec:dantzig-wolfe}, we
present the general theory of Dantzig-Wolfe decomposition for linear programs.
Section~\ref{sec:branch-and-price} then describes how to solve integer
programs by embedding the Dantzig-Wolfe decomposition within a
branch-and-bound framework, thereby obtaining a branch-and-price algorithm.

\section{The Dantzig-Wolfe Decomposition \wip}
\label{sec:dantzig-wolfe}

\subsection{Introduction}

The Dantzig-Wolfe decomposition is a decomposition technique that exploits the
block structure of large-scale linear programs. It was first introduced
by~\textcite{Dantzig_1960}. Formally, we consider problems of the form
\begin{subequations}
    \label{eq:dw}
    \begin{align}
        \min_{x^1,\dotsc,x^K} \quad & \sum_{k=1}^K {c^k}^\top x^k \\
        \text{s.t.} \quad & \sum_{k=1}^K A^k x^k \ge b, \label{eq:dw:coupling} \\
        & B^kx^k \ge d^k, \quad \text{for all } k=\dotsc,K. \label{eq:dw:blocks}
    \end{align}
\end{subequations}
Here, Constraints~\eqref{eq:dw:coupling} are considered to be complicating
constraints in the sense that, without them, the problem decomposes into $K$
independent subproblems of the form
\begin{equation*}
    \min_{x^k} \ \Defset{ {c^k}^\top x^k }{ B^kx^k \ge d^k }, 
\end{equation*}
for all $k=1,\dotsc,K$. For the sake of simplicity, we first assume that the
polyhedron $P^k \define \{ x : B^kx \ge d^k \}$ is actually a polytope, i.e.,
it is bounded. The goal of the Dantzig-Wolfe decomposition is to exploit this
block structure to solve the original problem more efficiently.

We start by reformulating Problem~\eqref{eq:dw}. In particular,
Constraints~\eqref{eq:dw:blocks} are equivalently expressed by representing
each $x^k$ as a convex combination of points~$\hat{x}^k$ satisfying $B^k
\hat{x}^k \ge d^k$ for each $k=1,\dotsc,K$. That is, we can write
\begin{equation*}
    x^k = \sum_{r \in \mathcal{R}^k} \lambda^k_r \, \hat{x}^k_r,
    \quad \sum_{r \in \mathcal{R}^k} \lambda^k_r = 1, \quad
    \lambda^k_r \ge 0 \quad \forall r \in \mathcal{R}^k,
\end{equation*}
where $\mathcal{R}^k$ denotes the set of extreme points of the $k$-th polytope
$P^k$, and $\lambda^k_r$ are the associated convex combination coefficients.

Substituting this representation into the objective function and the
complicating constraints~\eqref{eq:dw:coupling} yields the \emph{Dantzig-Wolfe
reformulation}:
\begin{subequations}
    \begin{align}
        \min_{\lambda} \quad & \sum_{k=1}^K \sum_{r \in \mathcal{R}^k} {c^k}^\top \hat{x}^k_r \, \lambda^k_r \\
        \text{s.t.} \quad & \sum_{k=1}^K \sum_{r \in \mathcal{R}^k} A^k \hat{x}^k_r \, \lambda^k_r \ge b, \\
        & \sum_{r \in \mathcal{R}^k} \lambda^k_r = 1, \quad \forall k=1,\dotsc,K, \\
        & \lambda^k_r \ge 0, \quad \forall k=1,\dotsc,K, \ r \in \mathcal{R}^k.
    \end{align}
\end{subequations}

In practice, the sets $\mathcal{R}^k$ are typically very large or even
exponential in size. Column generation is therefore used to iteratively add
only the columns (variables) corresponding to promising extreme points,
allowing the master problem to remain tractable while still converging to the
optimal solution of the original problem.

\subsection{Column Generation}

\subsection{Stabilization via Dual-Smoothing}

\section{The Branch-and-Price Algorithm \wip}
\label{sec:branch-and-price}

\subsection{The Algorithm}

\subsection{Hard vs.\ Soft Branching}

\section{Implementation}

\exsection{The Generalized Assignment Problem \wip}



\section{Strong Branching \wip}